\chapter{Design}

In this chapter, the process of deciding upon and making the design of the physical interface, will be explained. The design will be based on the formulated design requirements (see \autoref{sec:DRequirements}), as well as the common design principles of Gestalt\cite{gestalt}, the SOTA (see \autoref{sec:sota}), and knowledge gained from the workshop at Skt. Annæ Skole (see \autoref{sec:workshop}). 


\section{Initial design}
 The design could be taken in a broad variety of directions, and still live up to the requirements. As so, the design of the physical interface has been through lots of different concepts and iterations. 

\subsection {Workshop prototypes - The pre-initial designs}

During the analysis, a need for educational tools that were built with collaboration in mind, was discovered (see \autoref{sec:problemArea}). Based on that, multiple low fidelity concepts were created and brought to Sankt Annæ Skole for an ideation workshop. As mentioned in \autoref{sec:problemArea}, the prototypes at this point were used to discuss and discover new concepts and elements, which could be utilized in the final design. The findings from this workshop did not necessarily lead to new requirements, but instead served as pointers to which direction the design could be taken. The tool could make use of movement, which might conflict with or move the focus away from the learning aspect, and in such a case should be avoided. In another case, movement might serve to enhance the learning outcome (see \autoref{sec:interaction}), and should therefore be strived for. This however depends on the individual concept, and each finding from the workshop should therefore be discussed in relation to each design idea.\\

In order to evaluate upon many different elements and ways of collaborating and learning music, the aim of the concepts behind the prototypes, was to differ significantly from one another. Each concept will be briefly described in the following \autoref{fig:workshopPrototypes}.

\begin{figure}[H]
	\centering
	\includegraphics[width=0.9\linewidth]{figure/Design/workshopPrototypesDone} 
	\caption{Seen is the six design concepts which were presented at the workshop at Sankt Annæ Skole. Top left: a device that loops though entries(squares), and plays sounds if a box is placed on the entry point. Top middle: an ear training tool, where boxes labeled with tone names, should be orientated to display the heard tone. Top right: Wearables which produces individual sounds when shaken, and can be used to create and perform music. Bottom left: Board of a quiz game with music theory questions. Bottom middle: A set of joysticks to play instruments in a "band setting". Bottom right: A physical version of \textit{Garage Band}.}
	\label{fig:workshopPrototypes}
\end{figure}

\subsection{From workshop and requirements - the Crawford slip method}
To evaluate upon the workshop prototype concepts, in relation to the design requirements formulated and the knowledge gained from the workshop, a simplified version of the Crawford slip method was used (\autoref{sec:crawfordSlip}). By doing this, a list of suggestions to how elements could be used, and should not be used, was made and used as inspiration for other concept ideas. The full list can be seen in the appendix \autoref{CrawfordSlipList}.

With the project group divided into two groups of three persons, two design concepts were created (one for each group), with inspiration from this list. These two concepts, were then presented in the two groups, and discussed.

\subsubsection{Concept proposal 1: a variation of the \textit{"Simon Says"} game }
This design concept was highly inspired by the electronic game \textit{Simon says}\cite{simonSays}. Simon says is a musical memory game, where the player must repeat a sequence of tones generated by the game, by pressing the corresponding buttons on the interface. The idea was, that each student would have their own four push buttons, corresponding to the ones from the original game (see \autoref{fig:SimonSaysAlike}). The first player would then start the game, by choosing the first tone for the sequence, by selecting one of their buttons. The next player should then repeat the tone by pressing their equivalent button, and then add to the sequence by selecting the next tone. The players will continue taking turns until a player fail to repeat the sequence. This player will then no longer be part of the game, and must wait until the rest of the players have either failed to repeat the sequence, or won the game by being the last player left.

\begin{figure}[H]
	\centering
	\includegraphics[width=0.7\linewidth]{figure/Design/SimonSaysAlike} 
	\caption{Seen is the concept idea of a memory game alike \textit{Simon Says}. Each edge of the figure contains a set of four buttons, which belongs to a player. The players must take turns in memorizing, repeating, and then adding a new tone to the sequence, by pressing the buttons. Turn taking is visualized with LEDs}
	\label{fig:SimonSaysAlike}
\end{figure}

This concept relates to the study plan's section of \textit{Musical creation }(\ref{studyPlan}), by having the element of improvisation when having to pick the next tone. The concept may be used to play melodies together, or as a constructive competitive game. Either way, the students could learn to understand the tonal context between the four tones used in the game.   

\subsubsection{Concept proposal 2: Sequencer Mat}\label{sequencerMat}
This design concept revolved around the idea of a mat, which should function as a sequencer. The Mat should allow the user to actively perform sounds in the form of pure tones, by stepping on fields indicated as a grid on the mat. A sketch of the interface for this concept, can be seen in \autoref{fig:firstSketchOfMatFig}. Each field on the vertical axis, should produce a single pure tone which should be different from the others. These tones should relate to a scale, e.g. C,D,E,F,G,A,H. Each field along the horizontal axis, should produce the same pure tone. The sequencer functions could be: 
\begin{itemize}
    \item[-] Play
    \item[-] Record
    \item[-] Save
    \item[-] Add new
    \item[-] Reset
    \item[-] Different channels
    \item[-] Adjust pace
\end{itemize}
All of which should be controlled by buttons placed along the side of the mat, or near same.

\begin{figure}[H]
	\centering
	\includegraphics[width=0.9\linewidth]{figure/Design/firstSketchOfMat} 
	\caption{Seen is the first sketch of the interface for the sequencer-mat concept. In the middle of the figure, is a green 7x6 grid. Each field should activate a sound when stepped on. The horizontal axis of this grid, indicates time. The vertical axis indicates pitch of the sound. The pitch is also indicated by a color scheme along the same axis, gradually changing from a dark to a light color. This is illustrated in the first column of the grid. Along same axis are the tone names stated besides their given rows. To the left of the grid are 9 push buttons (drawn as circles) with different functions (indicated with symbols and text) placed vertical. }
	\label{fig:firstSketchOfMatFig}
\end{figure}

To clarify how the basic sequencer functionality (play and record) should function, an example of a use case has been made.
\paragraph{Example of an use case}
Three students want to play and record a set of tones which they want to consists of 3 tones with a break between each tone. They want to play the tones F, C and G, respectively, as seen in \autoref{fig:UseCase2}. The students place themselves on the mat, which the most left field activated being the tone F, then an empty column, then an activated field of the tone C, one more empty column, and lastly an activated field of the tone G. As so, it can be seen, that the tones are played from left to right.   
\begin{figure}[H]
	\centering
	\includegraphics[width=0.7\linewidth]{figure/Design/UseCase} 
	\caption{Shows an example of a use case of the sequencer. Three students are using the tool, recording the tones in the order of F, break, C, break, G. }
	\label{fig:UseCase2}
\end{figure}

To conclude on the two concepts; the sequencer mat concept included more of the desired elements from the workshop (see \autoref{sec:workshop}). As well as having a more clear relation to the study plan (see \autoref{studyPlan}) than the previous concept.
It was decided as being the definitive design concept.

\section{The Final design}\label{designConcept}
The concept, as explained in \autoref{sequencerMat}, was kept in the final design. However a lot of functionality and design was needed to be specified. These include the size, tempo, and musical scale of the mat, how the scale and time axis should be visualized, which functionality should be implemented, where they should be placed, and how they should be visualized.

\subsection{The setup}
The concept of the design includes visible electronic functionality. For the sake of securing and protecting the electronic components, a box was chosen as housing. As well as functioning as an interface for which the controls should be placed.

The placement of this control box in relation to the mat was chosen to be the most portable. As it can be seen in \autoref{fig:matVsBox}, three different solutions were discussed. A suggestion was to have the box fastened along one side of the mat. Another was to have it fastened in continuation of the middle column of the mat grid, and maybe have the mat constructed to be collapsible. Lastly, a suggestion was to make the mat and the box to be transported separately, and to be connected through a cable connector.     

\begin{figure}[H]
	\centering
	\includegraphics[width=0.7\linewidth]{figure/Design/maatteSetup}
	\caption{Shows the different proposals to how the mat and box should be connected. Top left is the box fastened along the edge of the mat. Top right is the box fastened to the top of one column of the mat, which is collapsible. At the bottom is the box and mat connected though a detachable cable}
	\label{fig:matVsBox}
\end{figure} 

The material for constructing the mat, was not yet chosen, which made it uncertain if fastening the box would be possible. For example if the mat was to be constructed by a flexible material, the connection joints between the mat and the box, might become overly exposed.
As the latter of the suggestions, where the mat and box was to be manually connected through a cable, did not depend on the choice of material, this design was chosen. 

\subsection{The size and sound of the mat } \label{sizeSoundColorMat}
The size of the mat affects the number of different tones it can produce, as the number of fields on the vertical axis is equivalent to the number of tones available. As this tool should be used to create and perform music in an educational context, using a musical scale would be appropriate in relation to learning scales. It would furthermore establish a hierarchy of the tones, due to the tonal context(which relates to the design principles of proximity and continuation\cite{gestalt}) present in a scale\cite{cognitiveFoundationOfPitch}. And thereby aid the student in understanding which field to stand on, to get the desired tone. \\
The most common and primitive scales, are the major and minor scales, also called \textit{natural scales}, which consist of 7 notes (C D E F G A H)\cite{scales}. One of these scales could have been chosen for this project, but the \textit{Pentatonic scale}, have a few advantages for which it instead was chosen. The Pentatonic scale consist of five of the notes from the natural scale (C D E G A), and has both the advantage of being able to be used in the same context as the natural scales, as well as sounding great no matter the combination of the notes\cite{pentatonicScale}. This makes this scale highly suitable for beginners and improvisational music\cite{pentatonicScale}.\\ 

With the pentatonic scale chosen for the tool, the vertical size of the mat was settled: 5 fields per row. The number of columns are equivalent to the beats of the sequencer function, and should therefore also reflect the musical aspect of tempo. A large variety of tempo could have been chosen, but as the most commonly used time is $\dfrac{4}{4} $\cite{tempo}, this was decided as the rhythm of the tool.\\\\
The size of the mat was decided to be 5x4 as seen on \autoref{fig:matSize}. As for the size of each field within the grid, the size of 27.5x27.5cm was chosen, based on the assumption that 4th graders have a shoe size ranging from size 35(EU) - 38(EU), which is approximately 22 - 25cm. This means that there will be room for one student per field.

\begin{figure}[H]
	\centering
	\includegraphics[width=0.8\linewidth]{figure/Design/finaldesign}
	\caption{Shows a 5x4 render of the mat, and the placement of the control box.}
	\label{fig:matSize}
\end{figure}

 To visualize the tonal context, ranging from a light to a dark tone, from top to bottom of the column, it was decided to use color to represent this transition. Four out of five of the discussed color schemes can be seen on \autoref{fig:colors}. The different color schemes discussed, were created with the intention of making a transition from a light to a dark color, resembling the tonal context. During the discussion, it became clear that the brightness of the colors, was perceived differently between individuals. To avoid confusion and ensure a common understanding for the transition, it was decided to use five different brightness levels of the same color. Two examples of this can be seen on \autoref{fig:colors}. The orange color scheme was chosen, as this is often refereed to as an energetic color\cite{orange}.

\begin{figure}[H]
	\centering
	\includegraphics[width=0.5\linewidth]{figure/Design/colors}
	\caption{Four different color schemes, for visualizing the tonal context.}	
	\label{fig:colors}
\end{figure}

To enhance the understanding of the columns as having the same tones, the design principle of similarity\cite{gestalt} was used, by using the same color scheme for each column, as seen in \autoref{fig:coloredMat}.\\\\

\begin{figure}[H]
	\centering
	\includegraphics[width=0.5\linewidth]{figure/Design/coloredMat}
	\caption{Shows a visualization of how the mat is colored.}	
	\label{fig:coloredMat}
 \end{figure}

\subsection{The control box}
The control box functions, as both protection for the hardware and as an interface for the controls of the sequencer. In this section, the thoughts concerning the dimensions of the box will be stated, and the functionality for the sequencer and the visualization of related buttons, will be discussed.

\subsubsection{The dimensions}
First of all, the box should be big enough to store the wires and components of the system. Besides this, the dimensions should take the use and possible misuse into consideration. With this, it is meant that actions such as using the controls on the box, should not make it fall over.\\\\
All of this taking into consideration, the dimensions of the box was decided to be, around the size of a A4 paper for the interface side, and taller than the mats, but still small enough to give the box a low center of gravity to prevent it from falling over.

\subsubsection{The controls} 
The sequencer functions such as play, record etc., should be settled upon in order to establish the overall functionality of the tool, the numbers of controls and their design. The discussed functions are:  

\begin{itemize}
	\item Play 
	\item Record 
	\item Delete
	\item Volume
	\item Octave up/down (this is to alter the tones pitch)
	\item Sequences
	\item Channels/tracks (To play from multiple sequences at the same time)
	\item Save/add
	\item time/pace/metronome (to alter the tempo)
	\item reset
\end{itemize}   

As the tools main purpose for this project, is to facilitate collaboration, it was decided to minimize the number of functions to keep the complexity of the tool to a minimum. The functions chosen was therefore limited to \textit{play}, \textit{record}, \textit{delete} \textit{four sequences}, and \textit{octave up/down}. \textit{Play}, \textit{delete} and \textit{record}, are the most basic functions of a sequencer.\\
The sequences functionality, was chosen in order to let the users make creations longer than four tones. The choice of having four sequences relates to the fact that music often builds upon even numbers, for which four is both one of the more common ones\cite{tempo}, and creates room enough for small melodies to be created.\\
The octave up/down was decided to be implemented due to the limited number of tones available when having five different fields. By being able to change the octave of the tones, more tones will be available, and therefore enable the possibility to create and replicate melodies which spans over multiple octaves (example of a such song: \textit{"frere jacques"}).\\\\
The icons for three most basic functions (play,delete,record) was solely based on the commonly used symbols for same functions (Used in for example \textit{Garage band}\cite{Garageband}).\\ Buttons for the four sequences was chosen to be visualized as numbers, as this signifies that the sequence of 1 should be recorded and played before sequence 2, and so on. To indicate active sequences, an LED for each should be lit if the related sequence is active. \\ The octave buttons was designed as arrows pointing in the up and down direction. The button designs can be seen in a render in \autoref{fig:buttonDesign}.\\\\
The placement of the buttons in relation to each other was settled upon, using the gestalt principles\cite{gestalt} as guidance. The three main buttons should be the same size, according to the principle of similarity\cite{gestalt}, and should be placed close together according to the principle of proximity\cite{gestalt}. The sequence buttons should be smaller and placed together following the same principles. The LEDs related to the buttons, should also follow the principle of proximity\cite{gestalt}.\\ The octave buttons follow same principles as the the other groups of buttons, size similarity, and proximity.
 

\begin{figure}[H]
	\centering
	\includegraphics[width=0.7\linewidth]{figure/Design/buttonDesign}
	\caption{The control buttons design.}
	\label{fig:buttonDesign}
\end{figure}


As the buttons were created in the following implementation phase, it was decided to change the color on the octave buttons from the neutral black, to a orange color gradient resembling the gradient seen on the mat. The idea was to indicate that the alteration made when pushing the buttons, was related to the tones on the mat. An indication which before were absent, as the black arrows, as well could have indicated a change in volume as in octave. A visualization of the buttons with the alteration in the octave buttons can be seen on \autoref{fig:buttonDesign2}.
\begin{figure}[H]
	\centering
	\includegraphics[width=0.7\linewidth]{figure/Design/buttonDesign2}
	\caption{The control buttons design with changed octave buttons.}
	\label{fig:buttonDesign2}
\end{figure}


By visualizing the tool with all of the design decisions made in this chapter, the final design should resemble \autoref{fig:designFinal}. 
\begin{figure}[H]
	\centering
	\includegraphics[width=0.7\linewidth]{figure/Design/DesignFinal}
	\caption{A visualization of the final design.}	
	\label{fig:designFinal}
\end{figure}

\section{Testing the design - Usability}\label{sec:designUsability}
To test the usability of the mat and the control panel on the box, a usability test was conducted. This section explains how the test was prepared and executed. Furthermore, the findings from the test will be described.

\subsection{Preparations of test}
The goal of the test was to determine how usable the prototype was for the users and to eliminate possible challenges that might arise while interacting with the system. By having the test participants complete a few tasks based on some of the expected use cases, the prototype was designed for. Observing how they manage to complete the tasks will indicate how the general usability of the system is and where the prototype is causing problems. To gather concrete data from the test, the System Usability Scale(SUS)\cite{susScale} was used. It is a 10-item Likert scale that is used to evaluate the usability of any given interactive system with both positively and negatively worded items. Based on prior research, item 8 of the SUS-scale was reworded so the word \textit{cumbersome} was replaced by the word \textit{awkward}\cite{susScale}.


\subsection{Location}
The test participants for the usability test were found by convenience sampling on campus on Aalborg University Copenhagen (AAU CPH) as mentioned in \autoref{chap:methods}. The participants were brought into a small room either, one person at a time or in pairs. The total number of participants was 10, 2 one-person tests and 4 pairs. The mat had been placed in the middle of the room and two computers were made available for the post test SUS-questionnaire. Furthermore, a moderator and two observers were present. In \autoref{fig:usabilityTest} the layout of the test is presented.

\begin{figure}[H]
	\centering
	\includegraphics[width=0.7\linewidth]{figure/Design/usability}
	\caption{Shows the test layout during the Usability test.}	
	\label{fig:usabilityTest}
\end{figure}

\subsection{Test procedure and observations}
The participants were briefly introduced to the mat and the control panel and were asked to explore the mat, to get a feel of the layout. During this exploration most of the participants quickly discovered the correlation between the color scheme and the tones. They were then asked to change the octaves and this caused some problems. The octave buttons were mistaken for volume buttons and therefore a lot of the participants would try to push the sequence buttons when asked to \textit{"octavate"}. When discovering the octave buttons, sometimes with help from the moderator, the next task was completed with more ease as they were asked to lower the octave. Afterwards, the participants were asked to record a short sequence. Every time they would press the record button and continue to walk freely around on the mat and try to stop recording by pressing the record button again. When asked to play back what they recorded they would discover that the recording did not sound anything like what they played as the mat functions like a sequencer which makes the mat only record one beat at a time was not understood. Finally, they were asked to make sense of the control panel and explain in their own words, what they would expect the different buttons did. The icons on the play, record, and delete buttons were self explanatory for most participants, however the sequence buttons were not. There were a lot of different explanations as to what the numbered buttons did e.g. add various filters and effects, change the octave, toggle buttons for enabling and disabling the columns of the mat as the number of columns and the number of sequences were the same. When all the tasks were completed some participants would comment on the lack of indication of which beat was playing when they were recording. They also requested indicators on the control panel to make it more clear what the buttons did. After the test, the participants were asked to fill out a SUS-questionnaire on one of the two available computers which was the final part of the usability test.

\subsection{SUS Questionnaire}
In \autoref{fig:susResults} the results from the SUS-questionnaire, that followed the Usability test, are visualized in the form of a box plot graph. The following is a list of the questions from the SUS scale. The odd-numbered items in blue are considered the positively worded questions, whereas the even-numbered items in red are worded negatively. This is relevant for later analysis of the results.\\

\begin{enumerate}
	\item  \textcolor{blue}{I think that I would like to use this system frequently.}
	\item  \textcolor{red}{I found the system unnecessarily complex.}
	\item  \textcolor{blue}{I thought the system was easy to use.}
	\item  \textcolor{red}{I think that I would need the support of a technical person to be able to use this system.}
	\item  \textcolor{blue}{I found the various functions in this system were well integrated.}
	\item  \textcolor{red}{I thought there was too much inconsistency in this system.}
	\item  \textcolor{blue}{I would imagine that most people would learn to use this system very quickly.}
	\item  \textcolor{red}{I found the system very awkward to use.}
	\item  \textcolor{blue}{I felt very confident using the system.}
	\item  \textcolor{red}{I needed to learn a lot of things before I could get going with this system.}
\end{enumerate} 
%SUS boxplot

\begin{figure}[H]
	\centering
	\input{figure/Design/sus.tex}
	\caption{Figure showing the results of the SUS-questionnaire. The question numbers on the x-axis correspond to the question numbers in the above list of SUS-items}	
	\label{fig:susResults}
\end{figure}

Due to the way the SUS-scale is constructed it is expected that the box + whiskerplots will alternate between high and low in \autoref{fig:susResults} if the usability is sufficient. In this case we see that the first question is low even though it should be high. This particular question was \textit{I think that I would like to use this system frequently} and since the test participants were made aware of the actual target group as being elementary school children learning music, the outcome of the first question is not surprising. Question 8, which was reworded to \textit{I found the system very awkward to use}, did not seem to cause any noticeable confusion as the participants filled out their questionnaire. However, results shows that there were mixed feelings towards the prototype's usage. This might be due to the fact that some of the pads on the mat started to behave more and more inconsistently throughout the day of the test.

\subsection{Findings}
To find out how the system scores on the SUS scale some calculations have been made. The scoring system of the SUS, ranges from 0-100 with increments of 2.5. The calculation is shown below in \autoref{eq:susCalc}:
\begin{equation} \label{eq:susCalc}
( (Q_1+Q_3+Q_5+Q_7+Q_9)-5+25-(Q_2+Q_4+Q_6+Q_8+Q_{10}) )*2.5
\end{equation}
\linebreak
To calculate the score 1 point must be subtracted by each odd-numbered item and the even-numbered questions must be subtracted by 5. This is then multiplied by 2.5 to get the final score. \autoref{tab:susScoreTable} shows the calculated scores for each participant and the overall mean of the SUS scores for all tests.

\begin{table}[H]
	\centering
	\caption{Table showing the SUS score for each individual test participants and the overall mean score. For full table see appendix \ref{appendix:susResults}}
	\label{tab:susScoreTable}
	\begin{tabular}{|c|c|l|l|}
		\hline
		Participant & SUS Score \\ \hline
		1           & 52.5      \\ \hline
		2           & 75.0      \\ \hline
		3           & 45.0      \\ \hline
		4           & 55.0      \\ \hline
		5           & 75.0      \\ \hline
		6           & 37.5      \\ \hline
		7           & 70.0      \\ \hline
		8           & 80.0      \\ \hline
		9           & 70.0      \\ \hline
		10          & 75.0      \\ \hline
		Total       & 63.5      \\ \hline
	\end{tabular}
\end{table}

\autoref{fig:susScore} shows how the overall test score for the Usability test performs on the full scale from 0-100. As seen in the figure the total score is somewhere in between \textit{OK} and \textit{GOOD}, which means that there is some room for improvement.

\begin{figure}[H]
	\centering
	\includegraphics[width=1\linewidth]{figure/Design/susScore}
	\caption{Figure highlighting the test score(63.5) on a scale representing SUS scores and their meanings\cite{susScore}}
	\label{fig:susScore}
\end{figure}

\subsection{Summary}
As seen in \autoref{fig:susScore} the mat has some challenges that should be fixed. The technical mat related issues are immediately apparent. Based on the observations and comments the two major problems were the missing button labels and the beat indicators.

\section{Improvements to the final design}\label{improvementsUsability}
The findings from the test was used to, not only trying to improve upon the design, but mainly to understand which parts of the design should be explained to more detail when conducting the upcoming final test, to limit the test being influenced by usability problems. The problem areas of the tool was attempted to be improved by putting on label stickers onto the box in proximity to the related buttons (this can be seen on \autoref{fig:finalbox1}), and by implementing an LED strip which should light up the edge of the mat, indicating each column of it, and light up the column where the beat is active, as can be seen on \autoref{fig:LEDstrip}.

\begin{figure}[H]
	\centering
	\includegraphics[width=0.7\linewidth]{figure/Design/LEDstrip}
	\caption{A visualization of the LED strip added to the design. Red colored LEDs separates the columns and yellow LEDs highlights the column for which the beat has reached when a sequence is played or being recorded}
	\label{fig:LEDstrip}
\end{figure}

This design iteration was not tested again after applying the changes, due to time limitations, and as so, both the problem areas concerning usability found from the test, and the functionality of the changes made, should be explained in detail, in order to limit the risks of usability related interference in the final test.
