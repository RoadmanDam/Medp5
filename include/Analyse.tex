\chapter{Analysis}\label{chap:analysis}
 
\begin{quote}
\end{quote}


\begin{itemize}\label{stuff}
	\item 
	\item 
	\item
\end{itemize}

\begin{figure}[H]
	\centering
	\includegraphics[width=0.9\linewidth]{figure/Analysis/skillslearn}
	\caption{textstuff}
	\label{fig:activelearn}
\end{figure}

\begin{enumerate}
	\item 
	\item
	\item
\end{enumerate}

\subsection{Workshop}\label{sec:workshop}
The workshop was conducted at Skt. Annæ Skole, which is not a regular public school. It does follow the general study plan of elementary school education, but it has a focus on musical education. Skt. Annæ Skole teaches musical concepts and performance on a higher level than a regular elementary school.\\

The goal of the workshop was to get an idea of what the children thought about their current music education, what tools they use and how these tools could be improved. The children were presented with different ideas for new and alternative tools that can be used in the education. The children were presented with six unique ideas, each based on an initial brainstorm. The ideas were drafted on the basis on the knowledge gathered in the early analysis, and were based on tools the children already used. The children had to provide feedback on the ideas and compare them to existing ideas, to find what elements of the different tools the students desire.\\

Across the different ideas in the workshop, a number of trends became apparent. The following points were the most prominent when it came to desired features:\\

\newpage
\begin{itemize}
	\item[-] \textbf{Movement}\\
	The children found the current education to be very inactive. They desired ways to be more active and have more movement as part of the education.\\
	\item[-] \textbf{Variation:}\\
	The children disliked using the same tools and same methods for learning various aspects. Having variation to keep things fresh would be more fun.\\
	\item[-] \textbf{Games:}\\
	The children love to learn through games. They feel like they remember a lot of what they learn, when they use games and they have fun while learning. \\
	\item[-] \textbf{Visuality:}\\
	The children asked for more visual components in their education. It is important to them that what they use are presented in front of them visually and had colors that were pleasing to look at.\\
	\item[-] \textbf{Physical:}\\
	The children loves when the tools are physical. If they can see, touch and feel the objects they use to learn, they have more fun and feel like they learn.\\
	\item[-] \textbf{Group work:}\\
	The children expressed great joy about working together with their classmates and collaborating in different ways about creating and understanding music.\\
	
\end{itemize}



\subsection{Target Group} \label{sec:targetgroup} 
Based on an analysis of how the current education is, and how the students perceive it, the target group for this project are children aged 8-12, currently attending a musical education in elementary school.\\

The music teachers will be considered a sub target group, as the tool would have to be used and explained by the teacher. The teacher would have to use the tool to provide knowledge and add meaning to what the tool represents.
 
\subsection{Learning tools used}%Mini SOTA
This section is about the learning tools that Hanna Jørgensen mentioned in the interview in \autoref{ProblemArea}.  

\subsubsection{GarageBand}
GarageBand is a software developed by Apple Inc, as seen in \autoref{fig:garageband}. It is used for allowing users to create, edit, and render music. The whole idea of GarageBand is that the software uses pre made MIDI keyboards, loops and arrays of different instrumental effects and voice recordings that can help the user further develop various sounds. This software is a very helpful tool for learning how to create music.
\begin{figure}[H]
	\centering
	\includegraphics[width=0.7\linewidth]{figure/Analysis/garageband}
	
	\caption{Shows the GarageBand's interface}
	\label{fig:garageband}
\end{figure}


\subsubsection{MuseScore}
MuseScore (seen \autoref{fig:MuseScore}) is an open-source software program for creating music notation. The main attraction of the software is the ability to play and practise sheet music anywhere. You are able to search for new music sheets and practice these by either listening to the notes or change the notation. It is available for windows, mac, IOS, Android, and Kindle Fire. It accepts an input via MIDI keyboard and can transfer to and from other programs via MusicXML, MIDI or others. This is a very useful tool for understanding and practise music notation. The program is build so that anyone is able to create music.

\begin{figure}[H]
	\centering
	\includegraphics[width=0.8\linewidth]{figure/Analysis/musescore.png}
	\caption{Shows a picture of sheet music that can be played or manipulated, to understand the music notation}
	\label{fig:MuseScore}
\end{figure}

\subsubsection{Go Fish - Music notation} 
Go fish is a card game played by two or more players. The rules state that five to seven cards are given from a standard 52-card deck to each player. The remaining cards are spread out on a table. When it is a players turn, that player can ask any other players if they have a card of a given value. If they do, that card is given to the player that asked and if they don’t the player is asked to “go fish” which means the player who asked, has to pick a card from the pile of cards on the table. To win is to gather as many sets of cards of the same value. Hanna from Skt. Annæ is playing this game with her students where she do not use the standard 52-card deck, but uses cards with music notation; Making it a learning tool for people to memorize or learning the different notations of music. Seen in \autoref{fig:gofish} is an example of go fish looks with notes.

\begin{figure}[H]
	\centering
	\includegraphics[width=0.7\linewidth]{figure/Analysis/gofish}
	\caption{An example that shows how Go fish can be played with notes}
	\label{fig:gofish}
\end{figure}

\subsubsection{Sibelius}
Sibelius is a software program for creating music notation. It was created by Sibelius Software. The software can edit, print, play the music using synthesized sounds, and produce scores. It can be used for playing the music or turning it into audio files. It supports basically any MIDI device and it also comes with some pre made sample files. In \autoref{fig:sibelius} a picture of what Sibelius looks like, can be seen. 

\begin{figure}[H]
	\centering
	\includegraphics[width=0.7\linewidth]{figure/Analysis/Sibelius}
	\caption{Shows what the software looks like with different notations}
	\label{fig:sibelius}
\end{figure}

\subsubsection{Clio Online} 
Clio Online is an online tool for students who wants to learn more about the different topics. You can learn topics such as Danish, English, German, religion, history, biology, sports, music, arts, food etc. This is a tool used by teachers to help their students with their studies. In every subject there are help tools such as videos, sound files and pictures to optimize each users' learning pattern. This online tool also provides the user the capability to choose what difficult level the material should be.

\subsubsection{Music work out}
Music work out is a physical tool-set developed by Anette Præst Nielsen, as seen in \autoref{fig:musicworkout}. It is a box consisting of different games or teaching methods for users who want to learn or have fun with music. There are currently two different editions; a teacher’s edition and a Game edition. The teacher’s edition contains around 800 cards and accessories. It is used for learning about music theory, notes- and ear training, rhythm cards, interval cards, rest cards, time signature cards, solmization cards, subdivision cards, clef card etc. The game edition contains 3 different games; Domino, Bingo, and Memory. Which also consist of different parts of the above topics.

\begin{figure}[H]
	\centering
	\includegraphics[width=0.7\linewidth]{figure/Analysis/musicworkout}
	\caption{An example of the teacher's edition for learning about music notation}
	\label{fig:musicworkout}
\end{figure}

\subsection{Choice of Direction}
The knowledge gained in the problem area chapter, provides a direction for the progress of the project. A look at the study plan gives an overview of how the schools currently teach music throughout the three areas of competence; Musical Creation, Musical Understanding and Musical Performance. An interview with Hanna Jørgensen provided insight into current issues with the musical education, which leads to an analysis of the potential target group.\\

With the target group established, an analysis of the tools currently being used by the target group, in the context of the information gathered throughout the interview and a workshop. This highlighted issues with a lack of physical tools, that the target group can use to learn music in collaboration with each other. The children in the workshop expressed issues with the lack of tools, that can be used when working in groups.\\

The following analysis will focus on the aspects found in the problem area. Collaboration and how working in groups affect learning will be researched.\\

\section{Working in groups - different aspects and approaches} % Sofie (sammenflet)
When working in groups, different aspects affect the productivity and level of performance and thereby affects the learning outcome for both the group and individuals\cite{GodKlassekultur}. In this section some of these aspects will be discussed. These include: The arrangement of groups, the teachers role, roles within a group, inclusion and exclusion from a group, feedback as an important group tool, and different collaborative learning methods such as cooperative Learning, constructive competition, and peer Learning.

\subsection{Group sizes}\label{GroupArrangement}
The number and sizes of the groups need to be taken into account, and correspond to the total number of students and teachers within the class. Assuming that there is a classroom with 32 students a grouping of 4 students in each group will mean that a teacher needs to prepare and manage 8 groups which can be more difficult. Fewer larger groups with more members, 6-8 students, however, will be more convenient for a teacher to manage, but the learning outcome of the individual students within the group might not be ideal\cite{collaborationSocialPedagogy}. The smaller the groups the better the cooperation and beneficial effects of competition can arise\cite{collaborationCompetitionGames}.

\subsection{The roles within a group}
Group stability and dynamics are other factors that teachers might need to consider. It might be more beneficial for the students to be arranged in groups based on the teachers prior knowledge about the students' behavior and stick to the same groups for longer periods\cite{collaborationSocialPedagogy}. This gives the group time to challenge the knowledge of the group, and learn the members differences and thereby gain understanding of the importance of different work approaches and processes\cite{laeringIPraksis}.

Differences among people when working in groups are important, as the group members contributes in different areas (academically and socially), and can thereby overlap each others strengths and weaknesses\cite{ProjektarbejdesKompleksitet}. These individual collaborative roles, inspired by Belbins' team roles\cite{ProjektarbejdesKompleksitet} can be categorized in three main categories: 
\begin{itemize}
	\item[-] The thinking role - focus is being creative, making ideas, and the outcome of the work. 
	\item[-] The doing role - focus is the process of the work (how and why).
	\item[-] The social role - focus is the well-being of the group. 
\end{itemize}

Having a representative from each category strengthens the group work and the result, and thereby the learning outcome\cite{ProjektarbejdesKompleksitet}. However, if not represented and the importance of each role is not acknowledged, it can lead to conflicts within the group. These conflicts, can often be seen as difficulties within certain areas i.e. problems with organizing, leading to unstructured work, or the feeling and expression of a not equal contribution, which can lead to social conflicts \cite{ProjektarbejdesKompleksitet}. 

\subsection{Social inclusion and exclusion}
Being a part of a group lies within human nature and is important for the individuals well-being as well as their motivation, and can therefore effect the learning process\cite{ProjektarbejdesKompleksitet}. Being excluded from a group has a negative effect on the individuals learning and social well-being. This can even influence the person being excluded to avoid future group work due to the psychological defeat. The Danish professor in social psychology Dorte Marie Søndergaard, calls this \textit{social exclusion anxiety}\cite{ProjektarbejdesKompleksitet}. Being excluded is often related to the contributions' value. If the group members' contribution does not meet the requirements or is seen as of less value, exclusion can occur. Exclusion can also be caused by character traits and personality, and might therefore not necessarily originate from academical skills\cite{ProjektarbejdesKompleksitet}.
To reduce the risk of exclusion and facilitate inclusion, the group must function socially and understand the importance of the different collaborative roles. The tasks that groups need to complete have to be designed in such a way that they encourage collaboration and discussion, and does not promote individual work, in order to produce an effective learning outcome\cite{collaborationSocialPedagogy}. To further enhance inclusion, feedback can be used. 


\subsection{Feedback}
Feedback given between members of the group, is a highly important part of group work\cite{laeringIPraksis}\cite{ProjektarbejdesKompleksitet}. It is with this tool that the group can reflect on their work, and together work towards a shared goal, improving the learning process and outcome for both group and individuals\cite{laeringIPraksis}\cite{ProjektarbejdesKompleksitet}. It is also a tool which can be used to acknowledge each others work, and enhance inclusion and social well-being\cite{laeringIPraksis}\cite{ProjektarbejdesKompleksitet}. 


\subsection{Collaborative learning Methods} 
When collaborating, the process and methods must relate to the type of work. Collaboration can therefore be conducted in many ways. In the following sections three acknowledged collaborative methods, namely \textit{Cooperative Learning, Constructive Competition, and Peer Learning}, will be discussed.  

\subsubsection{Cooperative learning}
A term which is used widely within the category collaborative learning(see \autoref{collabLearning}), is cooperation\cite{collaborationCooperation}. It is a teaching strategy in which children in smaller groups in a class room cooperate towards a common goal and thereby develop their social skills whilst building a common base of knowledge about a course subject\cite{collaborativeLearningTeachers}\cite[p.~15]{peerLearning}\cite{collaborationCompetition}\cite{cooperativeLearningPractice}. A teacher divides the children into smaller groups for either a long or short term period and assigns a task to them. The children are thereby responsible for each others learning, and cooperate towards the groups' success in solving the task. Meanwhile, the teacher walks around between groups and monitors their progress\cite{cooperativeLearningPractice}. Even though cooperative learning has shown positive results in learning outcomes of the children, it has also been criticized.In some cases, the children will be taking on a larger part of the work, while others do nothing. Cooperative learning has also criticized for the risk of causing competition between either group members or whole groups\cite{collaborationCooperation}.

\subsubsection{Constructive Competition}
Competition tends to be seen as a negative and destructive force when talked about in a learning context and is often conceptualized as an opposite to collaboration\cite{collaborationCompetition}. Humans compete to outperform others in various situations such as at work, in school or in games\cite{collaborationCompetitionGames}\cite{collaborationCompetition}. However, in the right conditions, it can be constructive as it is simultaneously a strong motivator in learning situations\cite{collaborationCompetitionGames} and makes children perform beyond their own expected abilities\cite{collaborationCompetition}. Competition is one of the core mechanics in video games and aside from  motivation, it increases excitement, involvement, attention, and if such a game is educational, these effects could potentially have positive effects on learning\cite{collaborationCompetitionGames}. 

\subsubsection{Peer Learning} %Sofie
In peer learning, equal individuals work together in peers to achieve their individual goals\cite{peerLearning}, often in a tutor-mentor or tutee-mentee relationship. The idea is, that through mutual help and support, knowledge should be shared between them, the goals will then be to either learn by teaching or by being taught. This can be done in different constellations\cite{collaborationCompetition}.

\begin{description}
	\item[Peer Instruction] The children arrive prepared to given course, the teacher asks a question about the subject, and the individuals state their answers. Next, the assigned peers discuss the question and based on this, state their reevaluated answers. In this process the students have prepared themselves about the same material, and so their knowledge might not vary significantly, and the roles of tutor and tutee might not be present.\\
	
	\item[Peer Tutoring] Each of the peers must prepare and be able to teach a given material, to another peer. The peers is therefore in the roles of \textit{tutor} and \textit{tutee}, and shifts roles after a certain amount of time. This process relies on the fact that each of the peers have a greater understanding of different subject than others, as the roles of tutor and tutee will become present.\\
	
	\item[Peer Mentoring] This resembles the process of peer Tutoring, but instead the students must have different skill and experience levels i.e. comes from two different grades. Instead of a tutor and tutee, the terms \textit{mentor} and \textit{mentee} is often used, as the social relationship is often more in focus, than achieving academically skills.\\
\end{description}   

For all different approaches to peer learning, it applies that the teacher must have a deep understanding of the individual students, both in terms of social and professional skills \cite{collaborationCompetition}. Otherwise the Peers might not have the right prerequisites for the collaboration\cite{collaborationCompetition}.


\subsection{Sub conclusion} %Sofie og Jens
Understanding how the teacher should form groups, and how the dynamics of the group affects the work, will help in understanding how a design strive to relate and support both these arrangements and collaborative roles. Inclusion will provide a higher learning outcome for both the group and for the individual. The collaborative methods can act as guidelines for the design and usage of the tool.\\

As most of the factors concerning group forming, dynamics, and arrangements,  will be in the hands of the teachers, they might not be something that can be directly designed for.  
However, based on the section's findings, the tool ideally could be used for groups of the size 2-6 students.
 

\newpage \section{State of the art - Design Inspiration}\label{sec:sota}
In this part of State of the art, the focus was on analyzing design inspiration. It provided insight into physical and digital interfaces. Each design inspiration uses at least one of the three learning methods; cooperative learning, constructive competition and peer learning, but will only state some of the connections each learning method provides.

\subsection{Guitar hero}\label{sec:guitarHero} 
Guitar Hero is a game developed for playing music with components that resembles instruments. The idea is that the player will see notes on the screen in different colors, seen in \autoref{fig:guitarHero}. Each color is represented on a controller that is shaped like a guitar.  The player then have to match and press the correct color on the controller, at the right time. Guitar Hero makes use of some aspects of constructive competition, due to having a point system where the better player would have more points. In addition, it also uses aspects from cooperative learning, by having shared health system where when played incorrectly, health will decrease and if it is depleted, all players lose.
 
\begin{figure}[H]
	\centering
	\includegraphics[width=0.7\linewidth]{figure/Analysis/guitarhero}
	\caption{The Guitar Hero interface}
	\label{fig:guitarHero}
\end{figure}

\subsection{Noteput} 
Noteput is an interactive music table with tangible notes, for learning notation. The idea is that you use the interactive screen that shows a staff which can be either a Treble Clef or Bass Clef. You can choose between the different notations and place them on the staff, to determine if they should be whole, half, quarter, eights, sharp, flat or natural notes, as seen in figure \autoref{fig:noteput}. The table also has an option to play or loop the sounds, so that the user can keep changing the sounds as the loop keeps playing. 

During the workshop is was established that the children wanted more movement, variation, gamification, visuality, and physicality which makes Noteput an inspiration for further development. The important parts of Noteput is the physical and tactile interface.

\begin{figure}[H]
	\centering
	\includegraphics[width=0.7\linewidth]{figure/Analysis/noteput}
	\caption{Shows a user placing tangible notes on the interactive screen}
	\label{fig:noteput}
\end{figure}

\subsection{Dato Duo} 
Dato Duo is a two person synthesizer. Dato duo is a combination of a synthesizer and sequencer for creating electronic music together. It can be played alone or with others. On one side it is build using the circular sequencer which loops the last eight notes that is played. On the other side are the controls for the synthesizer, which contains two sliders that controls the two digital oscillators and the filter-cutoff frequency. It can also be combined with MIDI components. Dato duo can be used for cooperative learning because the users will be making music together and helping each other in their task, as seen in \autoref{fig:datoduo}.

It can be played alone but is best used together with other people. The device focuses on giving each user different components to control. Each component helps the users combine sounds, and make the device more collaborative when making sounds together.
\begin{figure}[H]
	\centering
	\includegraphics[width=0.7\linewidth]{figure/Analysis/datoduo}
	\caption{Shows the Dato duo synthesizer being interacted with by two users}
	\label{fig:datoduo}
\end{figure}

\subsection{DropMix}
DropMix is a hardware instrument for music creation or mixing, and can be seen in \autoref{fig:dropmix}. It uses a large board with five different slots for placing cards with different values. It is operated by using a phone or tablet. There are three different modes, which are clash, party, and freestyle. Clash mode is a competitive game for two to four players. The goal of Clash is to be the first team to reach 21 points. Each card that is placed on the table scores a point. To replace a card in one of the slots, it must be either equal or a higher value than the card already in the slot. Each card has to be placed into matching colored slots. There are also multicolored wild-cards and black and white effect cards that can be placed onto any of the five slots. In party mode all players play together to mix music and score points. Freestyle mode is all about mixing the different cards to create various beats. Drop mix uses all three of the learning methods consistently, since there are three game modes that provide the users with the possibility of either learning from each other (peer learning), competing against each other (constructive competition) or trying to help each other in completing the required tasks (cooperative learning).

The importance of the device is the capability to change between playing together or against one another. It uses a board and cards which makes this playable anywhere since it does not take up much space.


\begin{figure}[H]
	\centering
	\includegraphics[width=0.7\linewidth]{figure/Analysis/dropmix}
	\caption{Shows the Drop mix package}
	\label{fig:dropmix}
\end{figure}


\subsection{Dance Dance Revolution}
Dance Dance Revolution(DDR) is a physical interactive dance platform created by the Japanese company Konami, and can be seen in \autoref{fig:dancedance}. Players will stand on the platform and step on the colored arrows on the stand of the platform, corresponding to the arrows that will be shown on the screen. DDR's design makes use constructive competition and cooperative learning.

From the workshop it was found that many of the children wanted movement. DDR incorporates movement and physicality, which can be connected to learning as mentioned in \autoref{interaction}.
\todo{and summarize section}
\begin{figure}[H]
	\centering
	\includegraphics[width=0.7\linewidth]{figure/Analysis/dancedance}
	\caption{shows the arcade version of the DDR controller}
	\label{fig:dancedance}
\end{figure}

\subsection{Just dance}
Just dance is a dance video game developed by Ubisoft Milan and Ubisoft Paris. The game was released mainly for usage on the Nintendi Wii. The whole idea about the game is that each user must mimic the motions of an onscreen dancer’s choreography. For each movement the user will get more points, and the more correct the movements are the more points the user will receive. A screenshot of the game can be seen in \autoref{fig:justDance}.

As the workshop concludes it is very important for the children to be able to move, and Just Dance adds the capability of moving freely around the room as long as the person is still within the range of the sensor.

\begin{figure}[H]
	\centering
	\includegraphics[width=0.7\linewidth]{figure/Analysis/justdance}
	\caption{Shows the games interface}
	\label{fig:justDance}
\end{figure}

\subsection{Gridi}
Gridi is a physical board, which functions as a sequencer. Gridi has a 16x16 grid of different holes, which can play sound, and can be seen in figure \autoref{fig:Gridi}. When a ball is placed in one of these holes it will be enabled. Gridi iterates over each line and plays sound according to the placement of the balls. Gridi is a cooperative tool, since multiple users can use it simultaneously.

\begin{figure}[H]
	\centering
	\includegraphics[width=0.7\linewidth]{figure/Analysis/gridi}
	\caption{Shows the Gridi interface, and the sequencer looping through the table.}
	\label{fig:Gridi}
\end{figure}

\subsection*{Sub Conclusion}
The state of the art will help shape the design requirements. They will work as inspiration for specific design features and implementations. The current tools that teach music, vary a lot in the aspects they focus on, some focus on playing, some on movement, some are based on a hardware product, and some on software. All of these individual products could be utilized as inspiration in the design phase.

\section{Analysis summary}
The initial analysis investigated and found that there is a consensus between learning music and developing other aspects of life. This led to further investigation in learning in general, where it was found that there are two main concepts of learning; passive and active learning. Active learning displayed a lot of positive results through interactions and emotions, which lead to research in these areas. In motivation theory, the concept of Cognitive Evaluation Theory(CET) was discovered. The CET argues that humans have more motivation working in groups, as they get social-contextual events, praise or other rewards. Furthermore, collaborative learning enhances various aspects of learning. Specifically, when it comes to music, collaborative work strengthens the understanding of concepts and enhances critical thinking.\\

In order to establish a target group, the study plan was investigated, and a music teacher was interviewed, to highlight issues with current musical education. It was found that there was a need for physical tools that elementary school students could use in collaboration with each other. A workshop was held with students at Skt. Annæ Skole, where several design and functionality ideas were presented. The feedback provided, gave an understanding of what kind of tool the students themselves wanted, on top of what the teacher had already provided as feedback. Therefore, the solution would be a physical tool, that the students could use in collaboration with each other in an elementary school setting.\\

Further investigation lead to a better understanding of the concepts of group work, group roles, and the three different collaborative learning methods; cooperative learning, constructive competition and peer learning.\\


\section{Final Problem Statement}\label{sec:FPS}
	How can a physical interface facilitate collaboration in small groups for elementary school children, aged 8-12 years in a musical education context?

	
\section{Design Requirements}\label{sec:DRequirements}

The following requirements, based on the analysis, are the physical framework that the design should strive to fulfill.
	\subsection*{Requirements}
		\begin{itemize}
			\item[-] It must accommodate at least 2 children.\\
			\item[-] It must accommodate max 6 children.\\
			\item[-] Must facilitate collaboration within at least one of the three musical areas of competence.\\
			\item[-] Must be designed for Children aged 8-12.\\
			\item[-] It should strive to not induce  negative emotions.\\
			\item[-] It should strive to induce positive emotions.\\
			\item[-] It must output sound.	\\	
			\item[-] It must be interactable.\\
			\item[-] It must have a physical interface.\\
		\end{itemize}
	
















		