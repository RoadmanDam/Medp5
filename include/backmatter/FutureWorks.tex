\chapter{Future Works}

\section{Usability}
This section will explore the possibilities to improve the usability test. After doing the usability test, an LED strip was added in response. No further usability test was done on the final iteration, with the new implementation to confirm that this change improved the usability of the design. For further development of the prototype, more usability tests would have to be done for each new iteration of the design or implementation.

\section{Test}
This section will focus on the future aspects of more or different tests of the final prototype.
The final test was divided into two parts, where the first part concentrated on asking the participants to do tasks, while being observed, by two observers taking notes on collaboration. The second part was a Likert scale questionnaire. Furthermore, Cronbach's $\alpha$ was used to analyze the Likert scale. For further development of the test a larger sample size, more questions, observations and time would be required. The Cronbach's $\alpha$ resulted in 0.5682 which is in fact a low score. This could be mitigated by having more than 10 questions, as a low amount of questions could cause extremes in the result\cite{likertItems}.\\ 
Another way of testing could be done using a T-test. A T-test compares two data sets, so for a T-test to work, the test would be conducted in a different setting. A test could for example be a comparison of The Music Mat with software, already being used by the target group, such as Sibelius. 

\section{Other Directions}
In this section, different directions will be explained in relation to which areas, the prototype could facilitate other than collaboration.\\\\

As mentioned in the analysis, when it comes to learning, music can help improve different basic skill-sets of a person such as cognitive skills. Besides collaboration, a big part of the prototype also focuses on the musical creation and the musical understanding, which is part of the EMU study plan\cite{studyPlan}. To test if the prototype could provide a learning improvement as well, a longitudinal test that will test the childrens' learning outcome over a longer period of time. Another factor to be considered is motivation. When conducting a short-term test, it can be difficult to test and conclude upon motivation. A user might find the prototype interesting and fun the first time they use it, hence they might feel motivated. With that being said, continuous test needs to be conducted to see if the user would use the prototype on a regular basis, or if they find the motivation to use it multiple times.

\subsection{Alternative Target Groups}
There is multiple alternative target groups which can be considered in relation to the usage of the prototype. The first one is the elementary school teacher. The prototype can cater to the teacher because, in the end, in an elementary school setting, the teacher's responsibility is to educate the children, and they have to create the tasks for the prototype. The prototype could be designed in a way that caters more for the teachers' needs rather than the students' needs, this would have to be investigated properly.\\

Another target group to be considered is private tutors, teaching small classes. Depending on what kind of music that is being taught at the establishment, the prototype could be used in different ways. One way could be to incorporate MIDI sounds into the prototype, which can then be used to teach students about different ways to create music  or different musical understanding techniques.