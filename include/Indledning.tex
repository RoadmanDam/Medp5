\chapter{Introduction}

%Sofies udkast til en introduktion
The Copenhagen Zoo, often simply refered to as \textit{Zoo}, is a Zoo with a mission to promote nature conservation by creating informative experiences for the park guests, concerning the animals in the Zoo. These experiences come in the shape of themed workshops, animal feedings and training lectures held by nature guides and zookeepers, enclosures which is made to resemble the animals' natural habitats and stimulate their natural behaviour, and through non interactive and interactive informative posters stationed at the enclosures. 

Zoo is also part of different research and conservational projects around the world, which is financed though the park's income. The main focus of these projects are behavioral biology, zoological medicine, and population biology. These areas are also the main focus for the conveyed information in the park.

The Zoo presented 3 potential problems to solve. These problems were not final, but worked as inspiration for this project. The problems were as follows: \\

"How do we make Zoo interesting in an all-year-around perspective using media technologies that can work with children, youngsters and adults?" \\

"Can we use media technologies to make Zoo so interesting for a 11-25 yrs. target group that they will keep visiting the Zoo?" \\

"Can Gamification and media technologies increase the level of interaction among children and youngsters both during and out o the season in the Zoo?" \\

The initial common trends in these problems was the use of media technology, to increase interactivity within the park. This is used to define the initial problem statement.\label{intro}
%As Zoos desired way of conveying information comes in form of experiences, it introduces a need for renewal to maintain the interest and to increase the amount of presentable information.

%Zoo conveys information through experiences. 

%A meeting with the Zoo provided the following insights. Zoo conveys information through experiences, and the duration which people stays is dependent on whether they believe they have experienced the park or not. Therefore it is in the interest of Zoo to have people stay longer - especially in areas that are less populated.
\section*{Initial Problem Statement}\label{sec:ips}
As so, the initial problem statement states: 

\textbf{How can a digital installation within \textit{Københavns Zoologiske Have} be interesting and informative to visitors of the park?}\  \todo{Ikke færdig, vend tilbage} 





%This has led us to the following initial initial problem statement:

%How do you enhance the experience of a place of public recreational entertainment, through the use of an interactive (audio) visual artifact?


%Facilitating a experience through the use of an visual artifact.

%Enhancing a novel experience through the use immerse interactive technology

