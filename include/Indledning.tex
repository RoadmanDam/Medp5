\chapter{Introduction}

%Sofies udkast til en introduktion
The Copenhagen Zoo \textit{Københavns Zoologiske Have}, often simply refered to as \textit{Zoo}, is a Zoo with a mission to promote nature conservation by creating informative experiences for the parks guests, concerning the animals in the Zoo. These experiences comes in form of themed workshops, animal feedings and training lectures held by nature guides and zookeepers, enclosures which is optimized to resemble the animals natural habitat and stimulate their natural behaviour, and through non interactive and interactive informative posters stationed at the enclosures. 
\\ Zoo is also part of different conservation and research projects around the world, which is financed though the parks income. The main focus of these projects are: behavioral biology, zoological medicine, and population biology. These areas is also the main focus for the dissemination in the park.  
\\ As Zoos desired dissemination comes in form of experiences, it introduces a need for renewal to maintain the interest and to increase the amount of presentable information. 
\\As so, the initial problem statement states: 

\textbf{How can an experience within \textit{Københavns Zoologiske Have} be interesting and informative to the parks guests?}   





%This has led us to the following initial initial problem statement:

%How do you enhance the experience of a place of public recreational entertainment, through the use of an interactive (audio) visual artifact?


%Facilitating a experience through the use of an visual artifact.

%Enhancing a novel experience through the use immerse interactive technology

