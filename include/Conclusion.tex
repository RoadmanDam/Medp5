\chapter{Conclusion}
This project aimed to create a physical interface, which should facilitate collaboration within small groups in a music educational context, and to evaluate upon the extend of collaboration. The design was based on findings from the analysis, which included both theory on learning in general, as well as an interview with an experienced music teacher, an ideation workshop with students. The design created was a pressure sensitive floor mat with a companion control box, which combined, acted as a sequencer. This design was usability tested to reduce the risk of usability related issues, which could influence the final test. Issues concerning usability, were taken into consideration in the execution of the finale test. The test was conducted at Skt. Annæ Skole, with groups of children from within the target group; 4th graders. The test was designed to measure, through observation and a Likert scale,  to what extend the physical interface constructed, would facilitate collaboration.\\\\
The test showed that, both observations and the measurements from the Likert scale, indicated that collaboration was present during the use of the tool. Due to a small sample size (N = 25) and the fact that groups were observed to use different methods of collaboration, including different levels of leadership, a comparison between the observed and measured data, could not clearly state how the tool affected the role forming, and how this affected the collaboration. However as the roles in form of leader roles and a box controller role was common for the test, it indicates that these roles might have a tendency to occur, when using the tool.\\\\
The test conducted was overall consistent and reliable. However, validity concerns, in form of a small sample size (N = 25) and missing external validity, due to only testing on one single class, and poor reliability due to a low Cronbach's $\alpha$ score.